%% LyX 2.1.4 created this file.  For more info, see http://www.lyx.org/.
%% Do not edit unless you really know what you are doing.
\documentclass[12pt,twoside,british,usenames,dvipsnames,hypens,final]{scrartcl}
\usepackage{amsmath}
\usepackage{amsthm}
\usepackage{fontspec}
\setmainfont[Mapping=tex-text]{TeX Gyre Pagella}
\setsansfont[Mapping=tex-text]{Gillius ADF No2}
\usepackage[a4paper]{geometry}
\geometry{verbose,tmargin=2.5cm,bmargin=2.5cm,lmargin=2cm,rmargin=2cm,headheight=3cm,headsep=0.5cm,footskip=2cm}
\usepackage{fancyhdr}
\pagestyle{fancy}
\usepackage{url}
\usepackage{setspace}
\setstretch{1.1}
\usepackage[unicode=true,
 bookmarks=true,bookmarksnumbered=true,bookmarksopen=true,bookmarksopenlevel=1,
 breaklinks=false,pdfborder={0 0 0},backref=false,colorlinks=false]
 {hyperref}
\hypersetup{pdftitle={Video Steganography using Motion Vectors},
 pdfauthor={E Liberis}}

\makeatletter
%%%%%%%%%%%%%%%%%%%%%%%%%%%%%% Textclass specific LaTeX commands.
\numberwithin{equation}{section}
\numberwithin{figure}{section}

%%%%%%%%%%%%%%%%%%%%%%%%%%%%%% User specified LaTeX commands.
%\raggedbottom                           % try to avoid widows and orphans
%\sloppy
%\clubpenalty1000%
%\widowpenalty1000%

\addtolength{\oddsidemargin}{6mm}       % adjust margins
\addtolength{\evensidemargin}{-6mm}

\usepackage{titlesec}
\titlespacing\section{0pt}{5pt plus 4pt minus 2pt}{1.5pt plus 0.5pt minus 0.5pt}
\titlespacing\subsection{0pt}{5pt plus 4pt minus 2pt}{0.5pt plus 0.5pt minus 0.5pt}

\setlength{\parskip}{0.5\baselineskip}
\setlength{\parindent}{0pt}

\usepackage{enumitem}
\setlist{nolistsep}

% Define macro \wordcount for including the counts
\hypersetup{%
    pdfborder = {0 0 0}
}

\usepackage{tikz}
\usetikzlibrary{positioning,arrows,fit}
\tikzset{
block/.style={
  draw, 
  rectangle, 
  minimum height=1.2cm, 
  minimum width=1.9cm, align=center
  }, 
line/.style={->,>=latex'}
}
%\usepackage{flafter}

\usepackage{gillius2}
\usepackage{fancyhdr}

\fancyhf[HLE,HRO]{\nouppercase{\leftmark}}
\fancyhf[HC]{\textsc{Part II Project Proposal}}
\fancyhf[HLO,HRE]{\thepage}
\fancyhf[FC]{}

\let\oldenumerate=\enumerate
\def\enumerate{
\oldenumerate
\setlength{\itemsep}{3pt}
\setlength{\parsep}{0pt}
}
\let\olditemize=\itemize
\def\itemize{
\olditemize
\setlength{\itemsep}{3pt}
\setlength{\parsep}{0pt}
}

\date{20th October 2015}

\makeatother

\usepackage{xunicode}
\usepackage{polyglossia}
\setdefaultlanguage[variant=british]{english}
\begin{document}
\bibliographystyle{plain}


\subject{\textsc{Computer Science Part II Project Proposal}}


\title{Video Steganography using Motion Vectors}


\author{E. Liberis, Homerton College\\
\texttt{\small{}el398@cam.ac.uk}}

\maketitle
\begin{center}
Project originator: E. Liberis
\par\end{center}

	\vspace*{0.5\paperwidth}

\begin{description}
\item [{Supervisor:}] Daniel Thomas
\item [{Director~of~studies:}] Dr. Bogdan Roman
\item [{Overseers:}] Dr. D J Greaves, Prof. J Daugman
\end{description}
\newpage{}


\section*{Introduction and Description of the Work}

Steganography refers to a set of techniques for concealing information
within seemingly innocent carrier data (covers), and for detecting
such hidden information. This is widely used for enabling communication
where the detection of the existence of the message would attract
unwanted attention~\cite{Government Policies}.

Files with commonly-found data formats are less likely to raise suspicion
as covers, and redundancy in the structure of the data provides opportunities
to hide data. Digital media tends to have both of these properties,
contributing to their widespread use as covers in earlier work~\cite{Image Steg,MV LSB,MV Phase,MV Steg}.
This project will be investigating the use of motion vectors in video
steganography.

Successive frames in a video often only differ by the movement of
pixels in some regions of the frame. This ``temporal correlation''
is exploited by modern video codecs, notably H.264, to improve compression
ratios by storing only the motion data since the last frame, where
doing so is cheaper than storing the whole frame.\footnote{Codec periodically inserts individually encoded frames (I-frames)
whenever picture changes significantly and at regular intervals (to
recover if a streaming error occurs).} Frames stored in this way are called P- or B- frames and are represented
by a list of blocks of pixels that have moved since the previous frame
(\textit{macroblocks}), and the direction of the movement (\textit{motion
vector})~\cite{H.264 Spec}. Computing motion vectors is a tradeoff
between accuracy and computation time, so codecs typically settle
for an approximation instead of an optimal MV. While MVs are encoded
losslessly, this accuracy tradeoff allows minor changes to them go
unnoticed~\cite{MV Phase}: a property essential to the steganographic
techniques underlying this project.

The main deliverable of this project will be a steganography tool
for embedding arbitrary data into H.264-encoded videos, using a variety
of motion-vector-related techniques. A common trick for reducing distortion
of the cover data is to modify the least-significant bits of encoded
values, be they LSBs of pixels in an image or values of a motion vector.
As well as na\"ively embedding the message in LSBs, there are some
more intelligent approaches for making messages harder to detect,
such as selecting target LSBs using a PRNG or avoiding regions with
low variance~\cite{Image Steg}. Methods that specifically target motion
vectors have been proposed by Zhang \emph{et al.}~\cite{MV LSB} and Fang
\emph{et al.}~\cite{MV Phase}, which make use of length, phase and inaccuracies
of a motion vector. 

The branch of steganography concerning the detection of embedded messages
is called steganalysis. Another deliverable of this project is to
provide a toolkit to perform video steganalysis that neatly integrates
with existing scientific computing packages (Matlab, Python-based
libraries). A popular approach in steganalysis is to look at statistical
differences of modified and unmodified values, as selectively modifying
certain values may make them stand out from the rest. General approaches
use histograms, Chi-Squared attacks~\cite{Image Steg}, and motion-vector-specific
attacks (Deng et al. ~\cite{MV Steg}), exploit correlations seen in
neighbourhoods of MVs and other statistical metrics. 

I propose a project, consisting of two main deliverables:
\begin{itemize}
\item Development of an application that allows end user to encrypt\footnote{Since properly encrypted data is indistinguishable from random noise
encrypting the secret message thwarts many statistical attacks which
would otherwise detect patterns in the data.} and embed their secret message into H.264 video files, using embedding
algorithms introduced above. 
\item Development of a steganalysis suite that implements some of the recent
research advances and integrates into scientific computation packages.
\end{itemize}

\section*{Resources Required}

I will be using C++ as a main language for developing the steganographic
tool, primarily due to my experience with it and the speed and ease
of interfacing with codecs: necessary for manipulating motion vectors.
Steganalysis tools will be developed using Matlab or Python, as both
have extensive support for scientific computation and are popular
within the community. 

I will be using my own computer, running Ubuntu 15.04, for convenience.
The project can be continued on MCS machines should it become unusable.
I accept full responsibility for this machine and I have made contingency
plans to protect myself against hardware and/or software failure.
Nightly backups will be made to a removable storage and \texttt{Dropbox},
and the code will be hosted on \texttt{GitHub} (with \texttt{git}
version control).

No other special resources are required. 


\section*{Starting Point}

This project uses some of the cryptography concepts introduced in
Part IB \textit{Security I} course. I have familiarised myself with
the general concepts of steganography and LSB embedding though some
introductory texts and relevant papers, and briefly looked at H.264
codec format. 

Implementing my own H.264 codec is out of scope for this project,
so a possibly modified version of a library (provisionally \texttt{libx264})
will be used to gain access to the motion vectors. Steganography tools
may also rely on some inbuilt routines of scientific computing packages
(plotting, histogram, matrices handling, etc.).

The project may also benefit from material in Part II courses \textit{Information
Theory}, \textit{Digital Signal Processing, \LaTeX~ and Matlab} and
\textit{Security II}.


\section*{Substance and Structure of the Project}

The aim of this project is to (a) develop a steganography tool, and
(b) develop a steganalysis suite, as described in the first section. 

Both aims will require further research into H.264 codec and ability
to modify MVs of a H.264-encoded video file, which would in turn require
parsing the format of the codec, modifying motion vector values, and
repackaging the data back into a playable video file. As developing
a H.264 codec does not relate to steganography and is potentially
error prone, I plan to leverage existing libraries and/or tools, such
as \texttt{ffmpeg}, to achieve this. As codecs are typically written
in C++, it is a natural choice for this task.

Modifying MV requires further research into steganographic techniques,
as well as the implementation and adaptation of these algorithms for
use in a codec. The application should also be able to take user data
and perform encryption prior to embedding, which would require using
a relevant cryptography library. 

The second objective will require substantial further research of
steganalysis methods described in recent academic papers, as well
as an investigation of how to use and integrate scientific packages
for implementing these methods. 

The dissertation shall consider the embedding techniques, evaluating
their relative applicability, detectability, embedding capacity, speed,
effectiveness, and resistance to attacks by developed steganalysis
methods. Detectability will also be evaluated by conducting a study
with human participants, asking them to distinguish between a modified
and an unmodified video. Effectiveness of steganalysis methods will
be evaluated by comparing their success in attacking implemented methods
on a dataset of videos.\footnote{Yet to be obtained, but that should not pose any problems.}

The steganography application will be manually and automatically tested
to ensure it works as reliably as expected. Individual modules will
be tested using unit tests and application as a whole using integration
tests. Development will follow the iterative (spiral) model, continuously
adding embedding techniques and steganalysis methods, and \texttt{gitflow}
workflow.


\section*{Success Criteria}

The project shall be considered successful if steganography application,
steganalysis tools and/or accompanying disseration (as appropriate)
satisfy the following requirements (prioritised using the \textit{MoSCoW}
rule):
\begin{itemize}
\item \textit{(M)} Ability to modify motion vectors of a H.264 video file.
\item \textit{(M) }Multiple LSB and non-LSB MV embedding techniques.
\item \textit{(M)} Multiple steganalysis methods.
\item \textit{(S)} Encryption of the secret message prior to embedding.
\item \textit{(M)} Explanation and justification of changes to or inapplicability
of some techniques, if required.
\item \textit{(S)} Compatibility with existing scientific computation packages
(\textit{Matlab} or Python-based packages)
\item \textit{(M)} Evaluation of the following properties of techniques:

\begin{itemize}
\item \textit{(M)} Applicability to MV steganography.
\item \textit{(M)} Resistance to attacks.
\item \textit{(C)} Embedding capacity.
\item \textit{(C)} Speed of processing videos.
\item \textit{(S)} Detectability by a human.
\end{itemize}
\item \textit{(M)} Evaluation of the following properties of methods:

\begin{itemize}
\item \textit{(M)} Effectiveness against implemented embedding techniques.
\item \textit{(S) }Usefulness (verbosity, amount of information provided)
to the steganalyst.
\end{itemize}
\item \textit{(M)} Introduction of the underlying theoretical background.
\end{itemize}

\section*{Extensions}
\begin{itemize}
\item Deng et al.~\cite{MV Steg} propose using statistical differences between
videos with and without embedded messages as features for a classifier.
Using this and other features proposed in similar works, a classifier
trained using machine learning could be produced.
\item Currently, social networks such as YouTube, Facebook and Tumblr transcode
(convert between formats) user-uploaded videos before presenting them
to other users. This results in MVs being recomputed and overwritten,
destroying the hidden message. More aggressive embedding techniques
could be developed, possibly using some control data and error-correction
codes, to withstand such transcoding. 
\end{itemize}

\section*{Timetable and Milestones}


\subsection*{Weeks 1 -- 2 (Oct 22 -- Nov 5)}
\begin{itemize}
\item Research and implementation of access and modification of MVs in a
H.264 video file.
\end{itemize}

\subsection*{Weeks 3 -- 4 (Nov 5 -- Nov 19)}
\begin{itemize}
\item Background reading on steganography and other relevant preparation.
\end{itemize}

\subsection*{Weeks 5 -- 6 (Nov 19 -- Dec 3)}
\begin{itemize}
\item Investigation and implementation of popular image steganography techniques,
including LSB embedding, and evaluation of their relevance to MV steganography.
\end{itemize}

\subsection*{Weeks 7 -- 8 (Dec 3 -- Dec 17)}
\begin{itemize}
\item Creation of tools that allow to carry out statistical attacks on aforementioned
methods and their evaluation.
\item Research and implementation of the methods of video watermarking proposed
by Zhang \emph{et al.}~\cite{MV LSB} and Fang \emph{et al.}~\cite{MV Phase}
\end{itemize}

\subsection*{Weeks 9 -- 10 (Dec 17 -- Dec 31)}
\begin{itemize}
\item Finishing implementation of aforementioned methods.
\item Christmas vacation and catch-up (if required).
\end{itemize}

\subsection*{Weeks 11 -- 12 (Dec 31 -- Jan 14)}
\begin{itemize}
\item Research and implementation of statistical steganalysis methods.
\item Evaluation of created tools against all implemented embedding algorithms.
\end{itemize}

\subsection*{Weeks 13 -- 14 (Jan 14 -- Jan 28)}
\begin{itemize}
\item Finishing any remaining implementation.
\item Writing progress report and preparing for the presentation.
\end{itemize}

\subsection*{Weeks 15 -- 16 (Jan 28 -- Feb 11)}
\begin{itemize}
\item Evaluation of the indistinguishability of videos with hidden messages
and ordinary videos using human test subjects.
\item Remaining evaluation tasks.
\end{itemize}

\subsection*{Weeks 17 -- 18 (Feb 11 -- Feb 25)}
\begin{itemize}
\item Any remaining catch-up work.
\item Implement extension tasks, if time permits.
\end{itemize}

\subsection*{Weeks 19 -- 20 (Feb 25 -- Mar 10) }
\begin{itemize}
\item Commence writing dissertation.
\item Ongoing evaluation tasks and further exploration of extension goals.
\end{itemize}

\subsection*{Weeks 21 -- 22 (Mar 10 -- Mar 24)}
\begin{itemize}
\item Write ``Introduction'' and ``Preparation'' chapters.
\item Request supervisors' feedback and iterate upon it.
\end{itemize}

\subsection*{Weeks 23 -- 24 (Mar 24 -- Apr 7)}
\begin{itemize}
\item Write ``Implementation'' chapter.
\item Request supervisors' feedback and iterate upon it.
\end{itemize}

\subsection*{Weeks 25 -- 26 (Apr 7 -- Apr 21)}
\begin{itemize}
\item Write ``Evaluation'' chapter.
\item Request supervisors' feedback and iterate upon it.
\end{itemize}

\subsection*{Weeks 27 -- 28 (Apr 21 -- May 5)}
\begin{itemize}
\item Finish remaining parts of the dissertation.
\item Polish and incorporate final feedback.
\item Submit the dissertation.
\end{itemize}

\begin{thebibliography}{1}
\bibitem{Government Policies} Omar J. Pahati \textit{``Confounding
Carnivore: How to Protect Your Online Privacy,''} AlterNet, Nov 28,
2001 (retrieved Oct 2015)\\
\url{http://www.alternet.org/story/11986/confounding_carnivore%3A_how_to_protect_your_online_privacy}

\bibitem{Image Steg}Philip Bateman \textit{``Image Steganography
and Steganalysis,'' }2008\\
\url{http://chemistry47.com/PDFs/Cryptography/Steganography/Image%20Steganography%20and%20Steganalysis.pdf}

\bibitem{MV Phase}Ding-Yu Fang; Long-Wen Chang, \textit{``Data hiding
for digital video with phase of motion vector},'' IEEE International Symposium
on Circuits and Systems, 2006. Proceedings.\\
 \url{http://ieeexplore.ieee.org/stamp/stamp.jsp?tp=&arnumber=1692862&isnumber=35661}

\bibitem{MV Steg}Yu Deng, Yunjie Wu, Linna Zhou, \textit{``Digital
video steganalysis using motion vector recovery-based features,''}
Appl. Opt. 51, 4667-4677 (2012)\\
\url{http://www.ncbi.nlm.nih.gov/pubmed/22781241}

\bibitem{MV LSB}Jun Zhang; Jiegu Li; Ling Zhang, \textit{``Video
watermark technique in motion vector,''} in Computer Graphics and
Image Processing, Proceedings of XIV Brazilian Symposium on ,
vol., no., pp.179-182, Oct 2001\\
\url{http://ieeexplore.ieee.org/stamp/stamp.jsp?tp=&arnumber=963053&isnumber=20786}

\bibitem{H.264 Spec}International Telecommunication Union (ITU),
\textit{``H.264: Advanced video coding for generic audiovisual services,''
}Feb 2014 (specification of the H.264 video codec)\\
\url{https://www.itu.int/rec/T-REC-H.264-201402-I/en}

\end{thebibliography}

\part*{\newpage{}Study involving Human Participants}


\subsection*{Description of the study}

The aim of this study is to determine whether using steganographic
methods to hide data in video files results in visible changes to
the video. The experiment will be conducted as follows. 

In each trial, participants will be shown pairs of ostensibly identical
videos, positioned side-by-side, one of which has been modified to
contain a hidden payload. They will choose the video which they think
was modified, after which the true answer will be revealed.

The experiment will feature 7-10 pairs of videos, each up to 30 seconds
long, taking up to 20 minutes per participant. Which video is the
modified one will be randomised, to avoid positional bias. Since the
steganographic methods rely on motion data encoded in the videos,
videos containing a wide range of movement, from almost still to highly
dynamic, will be used.

We hypothesise that human participants will not be able to distinguish
between modified and an unmodified video. To test this, we will determine
if the data support the opposite claim to a statistically significant
level -- that humans \textit{are} able to tell the videos apart. We
will look into the numbers of times the participants were right or
wrong in their choice and test whether there is a statistically significant
skew in either direction.


\subsection*{Precautions to be taken to avoid any risk}

No personal data about the participants will be collected and all
results will be recorded anonymously. Videos shown to the participants
will be free from flashing images, emotionally neutral, and free from
distressful, disturbing, offensive or otherwise unsettling images.\newpage{}


\part*{Resources Declaration}

The project does not require any special resources except own laptop,
for convenience.

Specifications:
\begin{itemize}
\item Intel(R) Core(TM) i5-3210M CPU @ 2.50GHz
\item nVidia Geforce GT 630M 2GB
\item 500GB HDD
\item 4GB RAM 
\end{itemize}
I accept full responsibility for this machine and I have made contingency
plans to protect myself against hardware and/or software failure.

The project can be continued on MCS machines should the laptop become
unusable. Nightly backups of the project files will be made to a removable
storage and \texttt{Dropbox}, and the code will be hosted on \texttt{GitHub}
(with \texttt{git} version control).
\end{document}
