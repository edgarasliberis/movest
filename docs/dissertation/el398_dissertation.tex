\documentclass[12pt,british,twoside,notitlepage,usenames,dvipsnames,hypens,final]{report}
%% Page setup
\usepackage[a4paper, twoside]{geometry}
\geometry{verbose,tmargin=3cm,bmargin=3cm,lmargin=2.5cm,rmargin=2.5cm,headheight=3cm,headsep=0.5cm,footskip=1.5cm}
\usepackage[unicode=true,
 bookmarks=true,bookmarksnumbered=true,bookmarksopen=true,bookmarksopenlevel=1,
 breaklinks=false,pdfborder={0 0 0},backref=false,colorlinks=false]
 {hyperref}
\hypersetup{pdftitle={Video Steganography using Motion Vectors -- CST Part II dissertation}, pdfauthor={E Liberis}}

\addtolength{\oddsidemargin}{6mm}
\addtolength{\evensidemargin}{-8mm}

\raggedbottom
\sloppy
\clubpenalty1000%
\widowpenalty1000%


%% Font and text flow setup
\usepackage{amsthm}
\usepackage{amsmath}
\usepackage{array}

\usepackage{polyglossia}
\setdefaultlanguage[variant=british]{english}

\usepackage{sectsty}
\allsectionsfont{\sffamily}

\usepackage{fontspec}
\setmainfont[Mapping=tex-text, Ligatures=TeX]{TeX Gyre Pagella}
\setsansfont[Mapping=tex-text, LetterSpace=1]{Gillius ADF}
\setmonofont[Mapping=tex-text]{Latin Modern Mono}

\usepackage{setspace}
\setstretch{1.1}

\setlength{\parskip}{0.5\baselineskip}
\setlength{\parindent}{0pt}

\usepackage{pifont}

%% List setup
\renewcommand\thesubsection{\arabic{subsection}.}
\usepackage{enumitem}
\setlist{nolistsep}
\setitemize{itemsep=2pt,topsep=0pt,parsep=5pt,partopsep=0pt}

%% Misc appearance things
\numberwithin{equation}{section}
\numberwithin{figure}{section}
\usepackage{multicol}
\usepackage{alltt}

\usepackage{titlesec}
\titlespacing\section{0pt}{4pt plus 0.5pt minus 0pt}{0pt plus 0.5pt minus 0.5pt}
\titlespacing\subsection{0pt}{5pt plus 4pt minus 2pt}{0.5pt plus 0.5pt minus 0.5pt}
\titlespacing\subsubsection{0pt}{5pt plus 4pt minus 2pt}{0.5pt plus 0.5pt minus 0.5pt}

\usepackage{epigraph}
\setlength{\epigraphrule}{0pt}

%% Some useful macros
\newcommand{\arr}{\textrightarrow\ }
\newcommand{\textsb}[1]{\textsf{\textbf{#1}}}
\newcommand{\textsbc}[1]{\sffamily \textsc{\textbf{#1}}}
\usepackage{lipsum}
\usepackage{tikz}
\newcommand*\circled[1]{\tikz[baseline=(char.base)]{
            \node[shape=circle,draw,inner sep=2pt] (char) {#1};}}

%%%%%%%%%%%%%%%%%%%%%%%%%%%%%%%%%%%%%%%%%%%%%%%%%%%%%%%%
\begin{document}

%% Title Page
\pagestyle{empty}

\hfill{\LARGE E Liberis}

\vspace*{60mm}
\begin{center}
\Huge
{\bf Video Steganography \\ using Motion Vectors} \\
\vspace*{10mm}
{ \sc \LARGE
Computer Science Tripos, Part II \\
Homerton College \\
}
\vspace*{10mm}
\the\year 
\end{center}

\cleardoublepage

%% Proforma
\setcounter{page}{1}
\pagenumbering{roman}
\pagestyle{plain}

{\section*{\Huge Proforma}}

{\large
\begin{tabular}{ll}
Name:               & \bf Edgaras Liberis                          \\
College:            & \bf Homerton College                         \\
Project Title:      & \bf Video Steganography using Motion Vectors \\
Examination:        & \bf Computer Science Tripos Part II, 2016    \\
Word Count:         & \bf XXXX\footnotemark[1]                     \\
Project Originator: & Edgaras Liberis                              \\
Supervisor:         & Daniel Thomas                                \\ 
\end{tabular}
}
\footnotetext[1]{This word count was computed
by {\tt detex diss.tex | tr -cd '0-9A-Za-z $\tt\backslash$n' | wc -w}
}
\stepcounter{footnote}
\vspace{0.5cm}

\section*{Original Aims of the Project}

The aim of this project is to implement and evaluate existing steganographic methods  applied to motion vectors. To achieve this an end-user tool should be developed, which integrates with a video encoding library and offers several embedding algorithms. These algorithms are compared based on several criteria, such as embedding capacity, speed, and detectability. To assess the latter a collection of general-purpose steganalysis routines should be created.   

\section*{Work Completed}

Applications for embedding and extracting data from motion vectors were developed, featuring several popular LSB / image steganography algorithms. Matlab functions and scripts for steganalysis were created to extract and analyse motion vectors, offering classic and motion-vector-specific attacks against embedding schemes. Algorithms were compared against each other evaluating detectability, capacity and other aspects. Experiment on Human Subjects was performed to evaluate visual footprint of data embedding.

\section*{Special Difficulties}

None.

\cleardoublepage

%% Declaration of Originality
\section*{Declaration of Originality}
I, Edgaras Liberis of Homerton College, being a candidate for Part II of the Computer Science Tripos, hereby declare that this dissertation and the work described in it are my own work, unaided except as may be specified below, and that the dissertation does not contain material that has already been used to any substantial extent for a comparable purpose.

\bigskip
\leftline{\bf Signed }

\medskip
\leftline{\bf Date}

\newpage
\tableofcontents

%% Chapters
\renewcommand{\thesection}{\arabic{chapter}.\arabic{section}}
\renewcommand{\thesubsection}{\arabic{chapter}.\arabic{section}.\arabic{subsection}}
\setcounter{chapter}{1}
\newcommand{\chapterheader}[2]{%
	\cleardoublepage
	\setcounter{chapter}{#1}
	\setcounter{section}{0}
	\chapter*{\scalebox{2}{\circled{#1}} \:\: #2 \hfill}
}
\setcounter{page}{1}
\pagenumbering{arabic}
\pagestyle{headings}

% Introduction
\chapterheader{1}{Introduction}
 
\emph{``Steganography''} is a compound word that comes from Greek \emph{``steganos''} (``covered, concealed, or protected''), and \emph{``graphein''} (``writing''). It is usually defined as an art of concealing information within seemingly innocent carrier data \cite[p. 3]{fridrich}. Steganographic methods are commonly used where communicating parties wish to hide the existence of messages between them, because that may attract unwanted attention. Such situations could be bypassing government censorship, avoiding law enforcement, and military intelligence, where the detection of the message may lead to revealing sender's location \cite{infohiding-survey}. 
 
\section{Motivation}

With the rise of the Information Age, modern steganography research has focused on exploring digital formats as containers for hidden data. Their appeal is due to redundancy in the structure of the data they encode, which often provides convenient opportunities to embed secret messages \cite[p. 2]{fridrich}. Another contributing factor is the increasing popularity of content-sharing websites, which made pictures, videos and audio recordings so widespread, that using them is unlikely to raise any suspicion. 

A sensible approach to digital steganography is to hide data within regions of a file that are highly tolerant to small modifications or noise. In images, for example, we could hide information by changing a pixel's colour to a very similar one. We can generalise this approach to so-called \emph{least-significant-bit (LSB) embedding} \cite{bateman}: typically the least significant bit in a binary representation of a value corresponds to the highest granularity level so changing it should not introduce a perceivable change.

Let's consider hiding information in a video file. To see whether video steganography could be feasible, let's do a high-level overview of a typical video file format. \texttt{MPEG} family -- a de facto industry and consumer standard -- encodes a video stream by encoding a sequence of frames that comprise it. However, encoding every single frame is redundant, because successive frames are often very similar (\emph{temporal correlation} property). Therefore some frames can be compactly represented as a set of changes from their predecessors. These changes are represented by how much a certain block has moved (\emph{motion vector (MV)}) and how pixels have changed (\emph{prediction error}) since the previous frame.  Frames encoded in full are called \emph{intra-frames} \cite{h264-std} (essentially a JPEG picture) and ones encoded as differences -- \emph{inter-frames} \cite{h264-std}. This gives us two avenues to explore: applying image steganography to intra-frames or using inter-frames' ``differences'' as a space for hidden data. Let's continue with the latter option, specifically considering embedding into MVs.

\textbf{PICTURE WILL BE GOOD HERE}

The amount of data one can hide -- embedding capacity -- is an important aspect of a steganographic system. Given that videos are essentially sequences of frames with sound, we expect them to have the largest embedding capacity out of all media containers. Let's do a rough estimate: there are typically $\approx$20 inter-frames per second and 1000-5000 usable motion vectors available per frame\footnote{Depends on the selection criteria and the resolution of a video file.}. This results in embedding capacity of $\approx$20-100 Kbits/s, which indeed very favourably compares against embedding data into image or audio files.

To be able to say how good a particular steganographic algorithm is, amongst other criteria, we need to evaluate whether the embedding it produces is detectable by an adversary. The counterpart field of study, which researches methods of detecting steganographic manipulations is called \emph{steganalysis}. Steganalysts use a multitude of domain-specific statistical attacks, such as plotting histograms, looking for correlations (or lack thereof) in the data, etc. to detect the presence of the hidden message. It is a cat and mouse game between the two fields, where new research is being published to break existing steganographic methods or the other way around -- to avoid detections.

Given all of above we can identify a niche for a project. This project explores inter-frame video steganography, specifically embedding data into MVs. To achieve this, LSB embedding is employed by allowing to tweak the $x$ or $y$ components of a vector. An end-user tool was developed, offering several high-capacity data embedding algorithms and encryption with a user-provided password. These algorithms are later compared based on several criteria, such as embedding capacity, speed and detectability, where latter is evaluated using general-purpose statistical steganalytic methods, that were implemented in Matlab. 

\section{Existing work}

Traditionally, image steganography has received more research attention compared to video steganography, so early attempts tried to straightforwardly use existing image (JPEG) steganography techniques \cite{bateman, jpegdctcoding}.

Bateman \cite{bateman} reviews the evolution of LSB-based embedding algorithms, starting with simple strategies, such as embedding data by changing the intensity level of every pixel. However, for data compression purposes images are often encoded using lossy JPEG, so these changes are lost after recompression. This can be mitigated by embedding data into something that JPEG stores directly in a compressed file, so later research focused on using DCT coefficients for this purpose\footnote{
One of the main compression techniques that JPEG uses is \emph{Discrete Cosine Transform (DCT)}, which is similar to the Fourier Transform. A relatively small amount of coefficients is enough to reconstruct an image with sufficient quality. Those coefficients are not sensitive to small changes, which means they are suitable for data embedding.} \cite{jpegdctcoding}. Other approaches were improving on this by preserving various statistical properties that ``clean'' images would possess \cite{bateman, f5}. These will be introduced in more detail further on. Similarly to this project, Williams \cite{scott-fs} applies these image steganography techniques to videos, but by considering uncompressed videos as series of JPEG-encoded frames.

Other researchers took a different approach and explored inter-frame steganography using motion vectors. Xu \emph{et al.} \cite{xu2006steganography} used a phase of a MV ($\tan^{-1}(\frac{y}{x})$) to determine whether the $x$ or $y$ component will carry a single bit of data (approach in this paper is implemented and analysed later in this document). Non-LSB algorithms include an interesting approach by Fang \emph{et al.} \cite{fang2006data}, who proposed to find an alternative motion vector (with minimum prediction error), whose phase will be in a particular quadrant out of 4. This conveys 2 bits of information per MV.

One of the main deliverables of this project is a usable end-user application for doing MV-based steganography, so I looked for competing applications in this area. While there are popular applications, such as MSU StegoVideo\footnote{\url{http://www.compression.ru/video/stego_video/index_en.html}} or OpenPuff\footnote{\url{http://embeddedsw.net/OpenPuff_Steganography_Home.html}}, they do not implement MV steganography, so evaluating them would inevitably lead to a discussion about the differences of theory behind them, as opposed to the quality and completeness of the implementation. The only application offering MV steganography that I was able to find is James Ridgway's \emph{Steganosaurus} \cite{steganosaurus}, but, unfortunately, I was not able to run it. The accompanying dissertation shined some light onto methods used, and it turned out that it only modifies the first (presumably the top left) motion vector of every frame, which severely limits the embedding capacity and increases chances of embedding being detected (location is known).   

Discuss steganalysis.

% Preparation
\chapterheader{2}{Preparation}

Outline, signposts

\section{Steganography Background}

Steganography definitions. Can make useful reference from RJA/MGK paper or Scott's dissertation or references he uses, or a book (Fridrich?). 

\section{Steganalysis Background}

Subsection describes what steganalysis is, its definitions and aims. This section will have more subsections. Can make useful reference from RJA/MGK paper or Scott's dissertation. 

\section{Video encoding background}

\subsection{Video encoding principles}

Subsection describes what video encoding is. What properties we exploit to efficiently encode videos? What approach is typically taken? (Try not to make this subsection too big)

\subsection{Motion Vectors as embedding space}

Subsection describes general idea how can we use motion vectors as a space for embedding data, arguing the undetectability. Give capacity estimates.

\section{Technical details}

Choice of tools, practical implementation, what approach could one take to implement this. (General approach, schematic, not the exact tools)

\section{Requirements analysis}

What do we want deliverables of the project to be? (described in the proposal)

\subsection{Steganographic Tools}

As above, specifically for encoder / decoder.

\subsection{Steganalysis Tools}

As above, specifically for steganalysis tools.

\section{Technical tools \& choices}

What technical decisions do we make before proceeding to implementation? Tools, languages, libraries? Justify everything.

\subsection{Steganographic Tool}

As above, specifically for encoder / decoder.

\subsection{Steganalysis Tool}

As above, specifically for steganalysis package.

% Implementation
\chapterheader{3}{Implementation}

Coherent summary of the following chapter, quickly outlining approach taken.

\section{Embedding and decoding process}

\subsection{Encoder}

How is encoder implemented? Structure, schematic, interesting parts. 

\subsection{Decoder}

How is decoder implemented? Structure, schematic, interesting parts. 

\section{Steganalysis techniques}

\subsection{Functions Supporting Extraction}

Typed / untyped MVs; sequential decoder

\subsection{LSB Plane + PLOT}

\subsection{Histogram}

\subsection{$\chi^2$ (Chi-Squared) Attack}

In general terms, don't tailor to MV case very much yet.

\subsection{Transcoding attack using a classifier}
SVMs are just the best.

\section{Embedding algorithms}

\subsection{Hide and Seek}

What's it about, outline basic concepts. Show histogram detectability.

\subsection{MSteg}

Why it avoids 0 and 1. Histogram motivation. Show equal pairs.

\subsection{F3}

Shirinkage. Effects of that. Histogram attack.

\subsection{F4}

How it avoids histogram attack? Chi-square attack on sequential attack.

\subsection{Randomised Hide-n-Seek}

Explain upsides of randomised embedding. Impl issues, mention ECC. 

\subsection{Outguess 0.1}

Similarly avoids 0s and 1s as most detectable ones. Current attack is JPEG specific?

\subsection{Cheng's Algorithm}

Discuss how MVSteg should be implemented + design decisions.

\section{Application}

\subsection{Design \& Usage}

Discuss implementation \& integration with libraries done.

\subsection{Encryption}
Discuss encryption features of the application.

\subsection{Initially proposed extensions}
Comment on why it didn't work out.

%**F5????**
%Minimising number of changes made to MVs

\section{Testing}

Discuss testing, manual / automatic.

% Evaluation
\chapterheader{4}{Evaluation}

Coherent summary of the following chapter, quickly outlining evaluation approaches.

\section{Evaluation of embedding and decoding process}

Say / evaluate what was done for embedding / decoding modules. Anything interesting? Difficulties encountered? Does app do what is expected, usability?

\section{Evaluation of embedding algorithms}

\subsection{Fixing Cheng's algorithm}

Dubious thresholding on length, see histograms. Which way to go?
Detectable angle thresholding. Bad vs good.
Detectable embedding 4x, Bad vs good.
Potential attack using revaqersion technique (and mention tools again here).

\subsection{Reversion Technique}

What is the result on the dataset? Could we make a primitive classifier? (Reference paper)

\subsection{Embedding capacity}

Talk about the differences (esp. randomised).

\subsection{Speed}

Mention how fast transcoding is

\subsection{Detectability by Humans (Study)}

\subsubsection{Hypothesis}

\subsubsection{Experimental procedure}

\subsubsection{Results}

Was the study hypothesis confirmed?

% Conclusions
\chapterheader{5}{Conclusions}

\section{Summary}

\section{Lessons learnt}

\section{Future directions}

People suggest different approaches, mention some.

\cleardoublepage
\bibliographystyle{unsrt}
\bibliography{refs}

\end{document}
